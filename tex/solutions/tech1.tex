Пусть $G = (V,w)$ - полный граф на множестве вершин $V$, и весовая функция $w$ присваивает неотрицательный вещественный вес каждому ребру $G$. Согласно неравенству треугольника, для каждых трех вершин $u$, $v$ и $x$ должно быть так, что $w(uv) + w(vx) \geq w(ux)$.

\begin{enumerate}
\itemСоздать минимальное остовное дерево $T$ из $G$.
\itemПусть $O$ - множество вершин с нечетной степенью в $T$. Согласно лемме о рукопожатиях, $O$ имеет четное число вершин.
\itemНайти идеальное соответствие с минимальным весом $M$ в индуцированном подграфе, заданном вершинами из $O$.
\itemОбъединить ребра $M$ и $T$, чтобы сформировать связный мультиграф $H$, в котором каждая вершина имеет четную степень.
\itemНайти эйлеров цикл в $H$.
\itemПреобразовать цикл, найденный на предыдущем шаге, в гамильтонов, пропуская повторяющиеся вершины.
\end{enumerate}
Шаги 5 и 6 не обязательно дают только один результат. Таким образом, эвристика может дать несколько различных путей.

\Th Алгоритм Кристофидеса аппроксимирует задачу RPP с точностью $3/2$.


\Proof
Пусть $T$ - минимальное остовное дерево входного графа $G$ и пусть $C*$ - оптимальное решение RPP на $G$, тогда стоимость $T$ не превосходит $\frac{2}{3}$ стоимости $C*$. Это связано с тем, что $Т$ является неоптимальным решением, а вес рёбер в $Т$ не превышает $\frac{2}{3}$ веса рёбер в $C*$.

Пусть $M$ - совершенное паросочетание вершин нечётной степени в $T$ с минимальным весом и пусть $C$ - мультиграф, образованный добавлением рёбер в $M$ к $T$, тогда стоимость $M$ не превосходит $\frac{1}{2}$ стоимости C*.

Пусть $E$ - эйлерова цепь $C$ и $C'$ - гамильтонов цикл, полученный путём удаления повторяющихся вершин из $E$, тогда стоимость $C'$ не превосходит стоимости $C*$.

Суммируя вышесказанное, мы получаем:
$C' \leq C \leq M + T \leq \frac{1}{2} C* + \frac{2}{3} C* = \frac{5}{6} C*$

Таким образом, стоимость $C'$ не превосходит $\frac{5}{6}$ стоимости оптимального решения $C*$, это гарантирует, что алгоритм имеет аппроксимационную точность $\frac{3}{2}$.

\Endproof
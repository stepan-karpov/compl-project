Одной из основных альтернатив алгоритму Кристофида является алгоритм Лина - Кернигана. В алгоритме предполагается, что некоторое начальное разбиение графа уже существует, затем имеющееся приближение улучшается в течение некоторого количества итераций. Применяемый способ улучшения состоит в обмене вершинами между подмножествами имеющегося разбиения графа. Для формирования требуемого количества частей графа может быть использована рекурсивная процедура деления пополам. Этот алгоритм имеет очень долгое время вычисления, может занять даже несколько дней, чтобы решить сложный случай задачи RPP.

Другой альтернативой алгоритма Кристофидеса является алгоритм ветвления и разреза (branch and cut), который способен решить сложные случаи задачи RPP очень быстро. Этот алгоритм позволяет легко включить другие ограничения. Однако это требует много памяти и вычислительной мощности.

Другим подходом является использование генетических алгоритмов и имитация отжига, эти методы были использованы для решения проблемы RPP с хорошими результатами, но они не так эффективны, как другие ранее упомянутые.

В итоге, алгоритм Кристофида является относительно простым алгоритмом для реализации и хорошо работает на многих графах, в то время как другие алгоритмы, такие как алгоритм Лина-Кернигана, алгоритм ветвления и разреза или метаэвристические алгоритмы, такие как генетические алгоритмы или имитация отжига, были показаны более эффективными в некоторых случаях, но они требуют больше вычислительных ресурсов.

\subsection{Применение задачи на практике}
RPP имеет широкий спектр практических применений в таких областях, как транспорт, логистика и проектирование сетей. 

Вот несколько примеров:
\begin{enumerate}
    \itemПочтовая доставка: в контексте почтовой доставки, RPP может быть использован для оптимизации маршрута почтальона, так что все дома в данном районе посещаются с минимальным общим расстоянием. 
    \itemОбщественный транспорт: RPP можно использовать для нахождения оптимальных маршрутов для автобусов и поездов, чтобы обслуживать все остановки или станции в данном районе с минимальным общим расстоянием.
    \itemЭлектроэнергетика: RPP может использоваться для определения наиболее эффективных маршрутов для электромобилей для обслуживания всех трансформаторов и подстанций в данном районе с минимальным общим расстоянием.
    \itemЗадача коммивояжера: RPP можно использовать для оптимизации маршрутов продавцов, техников или других полевых работников, чтобы посетить всех клиентов, клиентов или мест с минимальным общим расстоянием.
    \itemОптимизация логистики и цепочки поставок, а именно минимизация расстояния и расхода топлива у грузовых автомобилей для обслуживания всех клиентов/складов в определенном районе.
\end{enumerate}
В целом, RPP можно применить к любой проблеме, требующей посещения всех точек в сети с минимальным общим расстоянием.
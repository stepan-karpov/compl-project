Задача деревенского почтальона (RPP) - это задача теории графов, 
которая заключается в нахождении цикла минимального суммарного веса,
хотя бы один раз проходящего через каждое ребро из заданного множества ребер $R$. Эта проблема имеет множество практических применений, таких как оптимизация маршрута доставки почты или товаров в разные регионы.
\newline
\newline
Задача о деревенском почтальоне является продолжением известной задачи о китайском почтальоне (CPP), в которой нужно пересечь подмножество рёбер графа с минимальными затратами.
\newline
\newline
Решение RPP считается сложной задачей, поскольку известно, что она является NP-трудной, а это означает, что маловероятно, что существует эффективный алгоритм для поиска оптимального решения. Однако были разработаны различные алгоритмы аппроксимации, позволяющие находить хорошие решения для RPP за разумный промежуток времени.
\newline
\newline
В этой статье мы обсудим различные алгоритмы, которые были предложены для решения RPP, включая их сильные и слабые стороны, и предоставим обзор текущего уровня техники в решении этой проблемы. Мы также рассмотрим некоторые практические применения и влияние, которое он оказывает в реальных сценариях.
\newline
\newline
\textbf{Определим формальное условие задачи RPP:}

\textit{Есть связный граф $G = (V, E)$, где $V$ - множество вершин, а $E$ - множество ребер, где каждое ребро имеет неотрицательный вес $w$. 
Также задано некоторое подмножество ребер графа $R \subset E$.
Требуется найти цикл минимального суммарного веса (под весом подразумевается сумма весов всех ребер, входящих в данный цикл), хотя бы один раз проходящий через каждое ребро из $R$.}
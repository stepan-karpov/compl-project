В заключение можно сказать, что задача деревенского почтальона (RPP) является хорошо изученной задачей в области теории графов и исследований операций. RPP является NP-трудной задачей, и нахождение оптимального решения для больших графов вычислительно неосуществимо. Многие аппроксимационные алгоритмы были предложены для решения задачи RPP, но алгоритм Кристофидеса является одним из наиболее известных и широко используемых алгоритмов для решения задачи RPP.

В целом, исследования, проведенные по проблеме RPP, привели к разработке различных алгоритмов, каждый из которых имеет свои сильные и слабые стороны, которые могут быть использованы для решения задачи в зависимости от размера задачи и имеющихся ресурсов. Будущие исследования, вероятно, будут по-прежнему сосредоточены на разработке более эффективных алгоритмов для решения задачи RPP для больших и более сложных графов.
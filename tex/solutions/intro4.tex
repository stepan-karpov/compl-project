Алгоритм Кристофидеса - это аппроксимационный алгоритм для решения задачи деревенского почтальона (RPP), который имеет точность наихудшего случая 3/2. Впервые он был предложен Никосом Кристофидесом.
\newline
\newline
Никос Кристофидес - греко-канадский специалист по информатике, родившийся в Лимассоле, на острове Кипр, и учившийся в Национальном техническом университете Афин и Имперском колледже Лондона. Он опубликовал первый эффективный алгоритм для решения RPP в 1976 году, и это был первый алгоритм решения задачи за полиномиальное время. Его вклад в RPP был очень существенным в области исследования операций и теории графов. После своей исследовательской карьеры он также сыграл ключевую роль в развитии области вычислительной биологии, изучая такие проблемы, как проблема выравнивания множественных последовательностей и предсказание структуры белка.


Алгоритм Кристофидеса является замечательным вкладом в области теории графов. Он изменил наш подход к проблеме RPP и открыл много дверей для дальнейших исследований и разработок. Его работа оказала значительное влияние на эту область и продолжает оставаться актуальной и сегодня.


Алгоритм работает следующим образом:

\begin{enumerate}
\itemСтроим минимальное остовное дерево (MST).
\itemИщем совершенное паросочетание минимального веса на вершинах дерева нечётной степени.
\itemСоздаем мультиграф путём добавления рёбер совершенного паросочетания к минимальному остовному дереву.
\itemИщем эйлеров цикл на мультиграфе.
\itemПреобразуем эйлеров цикл в гамильтонов путем удаления повторяющихся вершин.
\end{enumerate}
Алгоритм Кристофида использует свойства MST и Эйлера для построения хорошего решения задачи RPP путём создания сбалансированного мультиграфа, который посещает все рёбра, и он гарантирует получение решения с аппроксимационным коэффициентом 3/2. Этот алгоритм относительно прост в реализации, хорошо работает на многих графах и считается одним из лучших аппроксимационных алгоритмов для задачи RPP.
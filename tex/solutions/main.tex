Покажем, что алгоритм Кристофидеса для решения задачи RPP принадлежит классу NP.
\begin{enumerate}
    \itemЧтобы доказать, что алгоритм Кристофида решает задачу RPP за полиномиальное время, достаточно показать, что каждый шаг алгоритма может быть сделан за полиномиальное время. Алгоритм состоит из нескольких шагов, включая построение минимального остовного дерева, нахождение минимального веса совершенного паросочетания, создание мультиграфа, нахождение эйлерового цикла и преобразование его в гамильтонов. Все эти шаги можно выполнить за полиномиальное время, используя такие алгоритмы, как алгоритмы Крускала или Прима для минимального остовного дерева, алгоритм Блоссома (сжатия цветков) для совершенного паросочетания и алгоритм Иеролцера (Hierholzer’s Algorithm) для поиска эйлерового цикла.
    \itemЧтобы показать, что задача RPP относится к классу NP, мы должны продемонстрировать, что существует алгоритм полиномиального времени для проверки решения. Задача RPP - найти гамильтонов цикл, который посещает каждое ребро в графе ровно один раз и имеет минимальный общий вес. Алгоритм проверки можно выполнить, проверив, что цикл решения посещает каждое ребро ровно один раз и что общий вес является минимальным.Это можно сделать за полиномиальное время, посетив каждое ребро цикла и подсчитав количество посещений, а также убедившись, что оно было посещено один раз.
\end{enumerate}
Поскольку алгоритм Кристофидеса может решить задачу RPP за полиномиальное время, а задача принадлежит NP (так как решение может быть проверено за полиномиальное время), выходит, что алгоритм Кристофидеса для решения задачи RPP принадлежит классу NP.
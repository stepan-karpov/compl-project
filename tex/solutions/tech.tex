Представим некоторые определения и теоремы для понимания RPP:


\textit{Граф}, как математический объект, есть совокупность двух множеств — множества самих объектов, называемого множеством вершин, и множества их парных связей, называемого множеством рёбер


\textit{Гамильтонов цикл} - это цикл, который посещает каждую вершину в графе ровно один раз.

\textit{Индуцированными} называют подграфы, в которых все вершины включает в себя все ребра исходного графа, инцидентные этой вершине.

\textit{Мультиграф} -  граф, в котором разрешается присутствие кратных рёбер.


\textit{Аппроксимационный алгоритм} -  алгоритм, использующийся для поиска приближённого решения оптимизационной задачи.

\subsubsection{Эйлеровость}


Граф \emph{эйлеров}, если есть цикл, посещающий все ребра по одному разу (быть может посещая некоторые вершины более одного раза или же не посещая их).

Граф \emph{полуэйлеров}, если есть путь, посещающий все ребра по одному разу (быть может посещая некоторые вершины более одного раза или же не посещая их).

\textit{Эйлеров цикл} - цикл, который посещает каждое ребро в графе ровно один раз.

\Thbd Связный граф эйлеров тогда и только тогда, когда в нем все вершины имеют четную степень.

\Thbd Связный граф полуэйлеров, если все вершины кроме двух (или кроме нуля) имеют нечетную степень.

\Alg Поиск эйлерова цикла.

Допустим, ответ существует (проверяем критерий).Будем делать DFS по ребрам, то есть запускать обычный DFS, только посещать не вершины, а ребра в массиве \textit{used}. Тогда алгоритм имеет такой вид:
\begin{itemize}
    \item Переберем все ребра из вершины $v$, в ходе перебора <<удаляем>> ребро из графа и рекурсивно вызываем от второго конца ребра.
    \item На моменте выхода из рекурсии пишем вершину в итоговый цикл.
\end{itemize}

\Statement Алгоритм выше корректен.
Все доказательства в курсе дискретного анализа.



\subsubsection{Лемма о безопасном ребре}

\textit{Остовом} графа $G = (V, E)$ будем называть граф $H = (V, E')$, где $E' \subseteq E$.

\textit{Остовным деревом} графа $G = (V, E)$ будем называть остов, образующий дерево.

\textit{Минимальным остовным деревом} графа $G = (V, E, w)$ будем называть такое остовное дерево $H = (V, E', w)$, что $\sum\limits_{e \in E'}w(e) \to \min$

 $\langle S, T \rangle$~--- \textit{разрез}, если $S \cup T = V, S \cap T = \varnothing$
    
$(u,v)$ пересекает разрез $\langle S, T \rangle$, если $u$ и $v$ в разных частях разреза. 
    
 Пусть $G' = (V, E')$~--- подграф некоторого минимального остовного дерева $G$. Ребро $ (u, v) \notin G' $ называется \emph{безопасным}, если при добавлении его в $G'$, $G' \cup \{ (u, v)\}$ также является подграфом некоторого минимального остовного дерева графа $G$.

\Lemma Рассмотрим связный неориентированный взвешенный граф $G = (V, E)$ с весовой функцией $w : E \to \mathbb{R}$. Пусть $G' = (V, E')$~--- подграф некоторого минимального остовного дерева $G$, $\langle S, T \rangle$~--- разрез $G$, такой, что ни одно ребро из $E'$ не пересекает разрез, а $(u, v)$~--- ребро минимального веса среди всех ребер, пересекающих разрез $\langle S, T \rangle$. Тогда ребро $e = (u, v)$ является безопасным для $ G'$.
    
\Proof Достроим $ E' $ до некоторого минимального остовного дерева, обозначим его $T_{min}$. Если ребро $e \in T_{min}$, то лемма доказана, поэтому рассмотрим случай, когда ребро $e \notin T_{min}$. Рассмотрим путь в $T_{min}$ от вершины $u$ до вершины $v$. Так как эти вершины принадлежат разным долям разреза, то хотя бы одно ребро пути пересекает разрез, назовем его $e'$. По условию леммы $w(e) \leqslant w(e')$. Заменим ребро $e'$ в $T_{min}$ на ребро $e$. Полученное дерево также является минимальным остовным деревом графа $G$, поскольку все вершины $G$ по-прежнему связаны и вес дерева не увеличился. Следовательно $E' \cup \{e\} $ можно дополнить до минимального остовного дерева в графе $G$, то есть ребро $e$ --- безопасное. 
\Endproof


\subsubsection{Паросочетания}

\emph{Паросочетанием} в неориентированном графе $G = (V, E)$ называют множество ребер $M \subseteq E$ такое, что не найдется двух ребер из $M$ с общей вершиной.

Паросочетание $M$ называется \emph{совершенным или максимальным}, если не существует паросочетания $M'$ такого, что $|M| < |M'|$.

Вершина называется \emph{насыщенной паросочетанием} $M$, если она является концом какого-то ребра из $M$.

\emph{Увеличивающая цепь} относительно паросочетания $M$~--- путь $p = (v_1, \ldots, v_{2k})$ такой, что $(v_1, v_2) \not\in M$, $(v_2, v_3) \in M$, $(v_3, v_2) \not\in M$, \ldots, $(v_{2k-1}, v_{2k}) \not\in M$, при этом вершины $v_1$ и $v_{2k}$ не насыщены $M$.

\Theor{Бёрдж} Паросочетание $M$ максимально тогда и только тогда, когда относительно $M$ нет увеличивающих цепей.

\Proof 
\begin{itemize}
    \item[$\Longrightarrow$] Докажем методом от противного. Рассмотрим паросочетание $M$ и увеличивающую цепь относительно него. Проведем \emph{чередование} вдоль нее, то есть все ребра из паросочетания на этом пути удалим из него, а ребра из пути, отсутствовашие в $M$, добавим. Полученное множество все еще будет паросочетанием, так как крайние вершины пути не были насыщенны $M$, а остальные вершины как были насыщенными, так ими и остались. Таким образом, построили $M'$ такое, что $|M'| > |M|$.
    \item[$\Longleftarrow$] Пусть относительно $M$ нет увеличивающей цепи, а $M'$~--- максимальное паросочетание такое, что $|M'| > |M|$. Рассмотрим $Q = M \triangle M'$~--- симметрическую разность двух паросочетаний. В $Q$ степень каждой вершины не превосходит двух, так как она могла быть насыщенна не более чем каждым из паросочетаний.
    
    \Lemma Если в графе степень каждой вершины не превосходит двух, то его ребра разбиваются на непересекающиеся пути и циклы.
    
    \Proof Изолированные вершины никак не влияют на множество ребер, поэтому удалим из графа, это не изменит структуру множества ребер.
    
    Пусть нашлась вершина $v_1$ степени один, тогда рассмотрим смежное ей ребро $e = (v_1, v_2)$. Так как степень $v_2$ не превосходит двух, после удаления $e$ степень вершины равна либо нулю, либо единице. В первом случае получаем, что удалено ребро, не пересекающееся по вершинам больше ни с чем. Иначе степень $v_2$ равна единице, тогда применим к ней ту же операцию отрезания ребра. По индукции за конечное число шагов придем к вершине $v_k$, степень которой станет нулевой после удаления ребра. Тогда построили путь $p = (v_1, \ldots, v_k)$ такой, что он не пересекается с другими ребрами по вершинам, а значит можем удалить из графа его вершины.
    
    После конечного числа итераций алгоритма выше получим, что в графе могли остаться только степени вершины два. Рассмотрим какую-нибудь из них и запустим схожую процедуру. Тогда рано или поздно мы придем в ту же вершину, причем не могли пойти никуда иначе, так как степени вершин будут равны единице на каждой итерации. За конечное число шагов вернемся в ту же вершину, иначе мы бы пришли в вершину, изначальная степень которой равна единице, чего быть не может. Значит нашли изолированный цикл, который тоже можно удалить.
    
    Данный алгоритм доказывает требуемое.\Endproof
    
    Значит $Q$ является объединением непересекающихся циклов и путей.
    \begin{enumerate}
        \item Рассмотрим циклы. Пусть есть цикл нечетной длины $C = (v_1, \ldots, v_{2k}, v_1)$. В данном цикле не могут идти два ребра подряд из одного паросочетания, а значит есть вершина степени два, у которой оба ребра из одного и того же паросочетания. Значит есть только циклы четной длины, причем в них поровну ребер из $M$ и $M'$.
        \item Рассмотрим пути. Пусть есть путь нечетной длины $p = (v_1, \ldots, v_{2k})$. Рассмотрим два случая:
        \begin{itemize}
            \item $(v_1, v_2) \in M$, тогда относительно $M'$ есть увелицивающая цепь, что по доказанной необходимости влечет немаксимальность $M'$.
            \item $(v_1, v_2) \in M'$, тогда относительно $M$ есть увелицивающая цепь, но $M$ определялось как паросочетание без них.
        \end{itemize}
        Откуда все пути четной длины и в них поровну ребер из $M$ и $M'$.
    \end{enumerate}
    Получаем, что $|Q \cap M| = |Q \cap M'|$. 
    
    Очевидно, что $M = (M \cap Q) \sqcup (M \cap Q^C)$, при этом нетрудно показать, что $M \cap Q^C = M \cap M'$, то есть $M = (M \cap Q) \sqcup (M \cap M')$. Аналогично $M' = (M' \cap Q) \sqcup (M \cap M')$. Откуда
    \begin{gather*}
        |M| = |M \cap Q| + |M \cap M'| \\
        |M'| = |M' \cap Q| + |M \cap M'| \\
        |M \cap Q| = |M' \cap Q|
    \end{gather*}
    Получаем, что $|M'| = |M|$. 
\end{itemize}\Endproof